\clearpage \noindent\begin{tabularx}{\linewidth}{|X|}
\hline \vskip -2mm
{\sf 信号与系统} \hfill August 29, 2011 \\
{\centering \sf \large Lecture 1:
绪论 \\ }
\textit{Lecturer: 王磊 \hfill Scriber: 戴唯思}\\ \hline
\end{tabularx}
\setcounter{section}{0}
\renewcommand{\thepage}{\lecture -\arabic{page}}
\def\lecture{1}

\section{授课人联系方式}

    \textbf{王磊}, 82668772-549, 西1楼 549办公室, \url{lei.wang@mail.xjtu.edu.cn}

\section{信号与系统}

    \begin{table}[h] \centering
        \caption{无处不在的信号与系统}
        \label{tab:1:signals-and-systems-everywhere}
        \begin{tabular}{ccc} \toprule
            信号 & 系统 & 响应 \\ \midrule
            语音 & 手机 & 电磁波 \\
            敲击 & 桌子 & 声响 \\
            油门上的压力 & 汽车 & 加速 \\
            \ldots & \ldots & \ldots \\
            \bottomrule
        \end{tabular}
    \end{table}

    信号的共性:

    \begin{itemize}
        \item 信号作为自变量的函数出现, 携带物理相关信息;
        \item 系统对信号作出响应, 产生一个或多个信号.
    \end{itemize}

\section{信息、消息、信号和系统}

    \begin{description}
        \item[信息] 消息中不确定性的度量
        \item[消息] 客观对象, 具体, 不便于远距离传输
        \item[信号] 消息的表现形式, 通常为物理量
        \item[系统] 产生、传输和处理信号的物理装置
    \end{description}

\section{信号与系统的用例}

    \begin{itemize}
        \item 系统辨识: 研究系统对给定输入信号所产生的输出响应\\\
            「信道估计」
        \item 系统设计: 信号处理和分析
        \item 信号设计
        \item 系统的反馈和控制
    \end{itemize}

\section{信号的分类}

    依据确定性可以分为确定信号(detrerministic signals, 可以根据数学表达式, 规则或表格确定的信号. 本课程只讨论确定信号)和随机信号(random signals, 表现存在不确定性, 无法被准确预测). 依据采样方式分为连续信号 (continuous-time signals)和离散信号(discrete-time signals). 此外, 还分为一维信号和多维度信号, 本课程只讨论一维信号.

\section{本课程的任务}

    \begin{enumerate}
        \item 通过信号分解学习确知信号的分析方法 \\
            时域, 频域, 变换域
        \item 通过信号分析学习分析LTI(Linear time-invariant)系统的方法
        \item 系统设计
    \end{enumerate}

\section{信号的表示}

    \begin{description}
        \item[连续信号] $x(t)$
        \item[离散信号] $x(n)$ or $x[n]$
    \end{description}

    共性: 表示为一元函数

\section{信号的能量与功率}

    \begin{description}
        \item[能量信号] 总能量无限: $E_\infty<\infty, P_\infty=0$
        \item[功率信号]     $E_\infty=\infty, 0<P_\infty<\infty$
    \end{description}
        
