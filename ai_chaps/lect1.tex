\def\lecture{1}
\clearpage \noindent\begin{tabularx}{\linewidth}{|X|}
\hline \vskip -2mm
{\sf 人工智能原理与技术(研)} \hfill September 6, 2011 \\
{\centering \sf \large Lecture \lecture:
绪论 \\ }
\textsl{Lecturer: 张选平 \hfill Scriber: 戴唯思}\\ \hline
\end{tabularx}
\setcounter{section}{0}
\renewcommand{\thepage}{\lecture -\arabic{page}}

\section{课程信息}

    课程重点介绍人工智能方法的一般性原理和基本思想, 包括问题求解的推理与搜索、计算智能、机器学习、Agent等技术和方法.

    参考教材: 
    
    \begin{enumerate}
        \item 蔡自兴, 人工智能及其应用:研究生用书3e, 清华大学出版社, 2004
        \item Russell.S, Norvig P., Artificial Intelligence: A Modern Approach, 清华大学出版社, 2006
        \item 鲍军鹏, 张选平, 人工智能导论
    \end{enumerate}

\section{绪论}

    \subsection{人工智能的定义与发展}

        \subsubsection{人工智能的定义}

            \textbf{智能}是一种认识客观事物和运用知识解决问题的综合能力, 包含感知能力, 记忆与思维能力, 学习与自适应能力, 行为能力.

            智能来源于思维活动, 取决于可以运用的知识, 可以由逐步进化来实现.

            人工智能目前尚无统一的定义. 一个定义如下: \textbf{人工智能}: 智能机器, 能够在各类环境中自主地或交互地执行各种拟人任务的机器.




    \subsection{人类智能与人工智能}

