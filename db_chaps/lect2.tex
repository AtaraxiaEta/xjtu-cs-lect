\clearpage \noindent\begin{tabularx}{\linewidth}{|X|}
\hline \vskip -2mm
{\sf 数据库系统原理} \hfill September 2, 2011 \\
{\centering \sf \large Lecture 2:
数据库系统 \\ }
\textsl{Lecturer: 冯中慧 \hfill Scriber: 戴唯思}\\ \hline
\end{tabularx}
\setcounter{section}{0}
\renewcommand{\thepage}{\lecture -\arabic{page}}
\def\lecture{2}

\section{数据管理各阶段的比较}

    \begin{itemize}
        \item 人工管理阶段
        \item 文件系统阶段
        \item 数据库系统阶段
    \end{itemize}

    \subsection{知名数据库产品}

        微软的有SQL Server, 此外还有开源的MySQL(先被Sun公司拥有, 然后被Oracle收购)

    \subsection{保证数据安全的方法}

        采用冗余备份, 记录log.

\section{数据库系统}

    \begin{figure}[h]\centering
        \begin{psmatrix}[rowsep=0.5]
            \psovalbox{程序员用户} \\
            \psovalbox{终端用户} & 三类用户 & \psdblframebox{数据库管理系统(DBMS)} & \psframebox{数据库(DB)} \\
            \psovalbox{数据库管理员(DBA)} \\
        \end{psmatrix}
        \ncangle[angleA=0, angleB=180]{-}{1,1}{2,2}
        \ncangle[angleA=0, angleB=180]{-}{2,1}{2,2}
        \ncangle[angleA=0, angleB=180]{-}{3,1}{2,2}
        \ncline{<->}{2,2}{2,3}
        \ncline{<->}{2,3}{2,4}
        \caption{数据库系统与用户}
        \label{fig:2:db-and-user}
    \end{figure}
   
    \subsection{数据模型与数据库模式}

        描述、组织和存放数据的方式涉及到数据模型和数据模式概念.

        \textbf{数据模型}(Data Model): 建立数据库的基础.

        \textbf{数据模式}(Data Schema): 在设计数据库的不同阶段需要建立不同的数据模式. 

        \subsubsection{数据和信息}

            数据在\textbf{语法}(Syntax)和\textbf{语义}(Semantic)两个层面上有意义, 语法指的是数据的格式规定, 语义指的是数据本身的含义.

            数据符号是对相应信息的抽象, 如: 将学号字段设计为字符串.

        采用不同的数据模型抽象描述现实世界中不同类型的事物.

        \subsubsection{数据模型}

            数据模型是规定现实世界数据特征的抽象, 是用来描述数据的语法、语义和操作的一组概念的集合, 包括\textbf{数据结构}、\textbf{数据操作}和\textbf{数据完整性}三要素.

            抽象的三个阶段:现实世界$\to$信息世界$\to$机器世界
            \begin{itemize}
                \item 概念数据模型: 面向用户和现实世界的数据模型, 与特定的DBMS无关
                \item 逻辑数据模型: 用户从DBMS看到的数据模型, 与DBMS相关
                \item 物理数据模型: 反映数据存储结构的数据模型
            \end{itemize}

            \begin{figure}[h]\centering
                \begin{psmatrix}
                 \psovalbox[fillcolor=lightgray, fillstyle=solid]{现实世界}\\
                 \psframebox{信息世界概念模型, 如E-R模型} \\
                 \psframebox{机器世界数据模型}
                 \end{psmatrix}
                 \ncline{->}{1,1}{2,1}\naput{认识抽象}
                 \ncline{->}{2,1}{3,1}\naput{转换}
             \end{figure}

        \subsubsection{数据模式}

            \begin{itemize}
                \item 外模式
                \item 模式
                \item 内模式
            \end{itemize}

