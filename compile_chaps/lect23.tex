\def\lecture{23}
\clearpage \noindent\begin{tabularx}{\linewidth}{|X|}
\hline \vskip -2mm
{\sf 编译原理} \hfill November 23, 2011 \\
{\centering \sf \large Lecture \lecture:
分层变量符号表 \\ }
\textsl{Lecturer: 冯博琴 \hfill Scriber: 戴唯思}\\ \hline
\end{tabularx}
\setcounter{section}{0}
\renewcommand{\thepage}{\lecture -\arabic{page}}

\section{符号表}

    \subsection{符号的作用域}

        \paragraph{分层变量符号表} 哈希链分为活名区和死名区, 元素指向变量信息四元组(变量名, 生存期, 同名next域). 5个表: 哈希表HT, 变量名串Char String, 符号信息表INFO, 符号表Symbol Table, 块表Block Table. 通过指针相联系. 常规动作: 填新项, 进入子程序, 查表, 退出子程序.

\section{运行时空间组织}

    分配策略: 动态分配, 静态分配. 允许递归的语言不能使用静态分配. 满足先申请后释放的用栈实现, 否则用堆实现.
