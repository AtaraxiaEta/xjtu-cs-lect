\def\lecture{16}
\clearpage \noindent\begin{tabularx}{\linewidth}{|X|}
\hline \vskip -2mm
{\sf 编译原理} \hfill October 26, 2011 \\
{\centering \sf \large Lecture \lecture:
控制语句中间代码的产生 \\ }
\textsl{Lecturer: 冯博琴 \hfill Scriber: 戴唯思}\\ \hline
\end{tabularx}
\setcounter{section}{0}
\renewcommand{\thepage}{\lecture -\arabic{page}}

\section{语义分析和中间代码的产生}

    \subsection{控制语句的翻译}

        \subsubsection{标号}

           标号允许先使用再定义.

            \paragraph{定义}

                符号表: 名字, 类型(标号), \ldots, 定义否(先使用的情况就写没有定义), 地址(定义的地址或调用的地址, 使用拉链法).

            \paragraph{使用}

                无条件转移, 根据名字查表.

            \paragraph{出错}

                若已经在符号表中但重复定义或原类型不是标号则出错.

        \subsubsection{条件语句}

            条件语句可以嵌套.

            \paragraph{解决办法}

                给终结符附带语义项Chain.
